% Created 2019-09-22 Sun 17:20
% Intended LaTeX compiler: pdflatex
\documentclass[nobib]{tufte-handout}


\usepackage{minted}
\usepackage{ifluatex, ifxetex}
%Next block avoids bug, from http://tex.stackexchange.com/a/200725/1913
\ifx\ifxetex\ifluatex\else
\newcommand{\textls}[2][5]{%
\begingroup\addfontfeatures{LetterSpace=#1}#2\endgroup
}
\renewcommand{\allcapsspacing}[1]{\textls[15]{#1}}
\renewcommand{\smallcapsspacing}[1]{\textls[10]{#1}}
\renewcommand{\allcaps}[1]{\textls[15]{\MakeTextUppercase{#1}}}
\renewcommand{\smallcaps}[1]{\smallcapsspacing{\scshape\MakeTextLowercase{#1}}}
\renewcommand{\textsc}[1]{\smallcapsspacing{\textsmallcaps{#1}}}
\fi
\usepackage{fontspec}
\setmainfont{ETBookOT}
\setmonofont[Scale=0.6]{Fira Code Regular}
\renewcommand{\contentsname}{contents}
\titleformat{\chapter}%
[display]% shape
{\relax\ifthenelse{\NOT\boolean{@tufte@symmetric}}{\begin{fullwidth}}{}}% format applied to label+text
{\huge\thechapter}% label
{0pt}% horizontal separation between label and title body
{\huge\rmfamily}% before the title body
[\ifthenelse{\NOT\boolean{@tufte@symmetric}}{\end{fullwidth}}{}]% after the title body
\titleformat{\section}%
[hang]% shape
{\normalfont\Large}% format applied to label+text
{\thesection}% label
{1em}% horizontal separation between label and title body
{}% before the title body
[]% after the title body
\titleformat{\subsection}%
[hang]% shape
{\normalfont\large\itshape}% format applied to label+text
{\thesubsection}% label
{1em}% horizontal separation between label and title body
{}% before the title body
[]% after the title body
\renewcommand{\maketitle}{%
\begingroup
\setlength{\parindent}{0pt}%
\setlength{\parskip}{4pt}%
\LARGE\scshape\plaintitle\par
\Large\itshape\plainauthor\par
\Large\itshape\thedate\par
\endgroup
%\thispagestyle{plain}% suppress the running head
%\tuftebreak
%\@afterindentfalse\@afterheading% suppress indentation of the next paragraph
}
\author{james gilles}
\date{september 17 2019}
\title{18.337 homework 0}
\hypersetup{
 pdfauthor={james gilles},
 pdftitle={18.337 homework 0},
 pdfkeywords={},
 pdfsubject={},
 pdfcreator={Emacs 26.3 (Org mode 9.2.6)}, 
 pdflang={English}}
\begin{document}

\maketitle
\tableofcontents


\section{problem 1}
\label{sec:orgb10beff}

This was fairly straightforward; i just followed the instructions in the README.
Output:
\begin{minted}[frame=lines,linenos=true,mathescape]{julia}
$ sbatch submit_forkjoin.sh
$ cat top5norm_forkjoin.log-442748
connecting to worker 1 out of 13
connecting to worker 2 out of 13
connecting to worker 3 out of 13
connecting to worker 4 out of 13
connecting to worker 5 out of 13
connecting to worker 6 out of 13
connecting to worker 7 out of 13
connecting to worker 8 out of 13
connecting to worker 9 out of 13
connecting to worker 10 out of 13
connecting to worker 11 out of 13
connecting to worker 12 out of 13
connecting to worker 13 out of 13
Added workers: 13
Pair{String,Float64}["cavorite"=>1.0; "mooncalf"=>1.0; "lunar"=>1.0; "cavor"=>1.0; "selenites"=>1.0]
Pair{String,Float64}["leblanc"=>1.0; "holsten"=>1.0; "fowler"=>1.0; "brissago"=>1.0; "karenin"=>1.0]
Pair{String,Float64}["montgomery"=>1.0; "puma"=>1.0; "moreau"=>1.0; "prendick"=>1.0; "swine"=>0.964286]
Pair{String,Float64}["withered"=>0.692308; "candles"=>0.619048; "candle"=>0.567568; "shade"=>0.296296;
                     "room"=>0.0686275]
Pair{String,Float64}["psychologist"=>1.0; "morlocks"=>1.0; "weena"=>1.0; "filby"=>1.0; "sphinx"=>1.0]
Pair{String,Float64}["artilleryman"=>1.0; "ulla"=>1.0; "martian"=>1.0; "woking"=>1.0; "martians"=>0.993939]
Pair{String,Float64}["henfrey"=>1.0; "adye"=>1.0; "mariner"=>1.0; "griffin"=>1.0; "jaffers"=>1.0]
Pair{String,Float64}["vee"=>1.0; "ann"=>1.0; "alice"=>1.0; "miniver"=>1.0; "veronica"=>1.0]
Pair{String,Float64}["graham"=>1.0; "aeropile"=>1.0; "howard"=>1.0; "asano"=>1.0; "isbister"=>1.0]
Pair{String,Float64}["smallways"=>1.0; "edna"=>1.0; "germans"=>1.0; "bert"=>1.0; "butteridge"=>1.0]
Pair{String,Float64}["boomfood"=>1.0; "cossar"=>1.0; "hickleybrow"=>1.0; "wondershoot"=>1.0;
                     "herakleophorbia"=>1.0]
Pair{String,Float64}["dad"=>1.0; "slim"=>1.0; "industrialist"=>1.0; "explorer"=>0.956522;
                     "astronomer"=>0.953488]
Pair{String,Float64}["polly"=>1.0; "johnson"=>1.0; "rumbold"=>1.0; "fishbourne"=>1.0; "parsons"=>1.0]
\end{minted}

\section{problem 2}
\label{sec:org4af8ee4}

this was fairly straightforward.

\begin{minted}[frame=lines,linenos=true,mathescape]{julia}
println("booting julia...")
using ClusterManagers, Distributed
using DataFrames, CSV
println("done.")

println("demanding workers from slurm...")
addprocs(SlurmManager(parse(Int,ENV["SLURM_NTASKS"])-1))
println("done. added workers: ", nworkers())

println("prepping computation...")
@everywhere begin
    using Random

    # brownian motion
    make_brownian(dt,n) = cumsum([0;sqrt(dt).*randn(n+1)])

    # get the walk for some seed, and grab the hostname as well
    get_results(seed) = begin
        Random.seed!(seed)
        brownian = make_brownian(1, 100)
        hostname = gethostname()
        (brownian, hostname)
    end
end
println("computation prepared.")

println("running...")
seeds = 1:50
results = pmap(get_results, seeds)
println("done.")

println("writing output...")
function format_results(results :: Array{Tuple{Array{Float64, 1}, String}, 1})
    # bundle all outputs into a single dataframe
    all = DataFrame(time = [], y = [], hostname = [])
    for (walk, hostname) in results
        current = DataFrame(time = 1:length(walk), y = walk, hostname = [hostname for i in 1:length(walk)])
        append!(all, current)
    end
    all
end

CSV.write("results.csv", format_results(results))
println("done.")

\end{minted}




\begin{minted}[frame=lines,linenos=true,mathescape]{julia}
using Gadfly
using DataFrames

const steps = 100

make_brownian(dt,n) = cumsum([0;sqrt(dt).*randn(n+1)])

function render_results(data :: Array{Tuple{Array{Float64, 1}, String}, 1})
    all = DataFrame(time = [], y = [], hostname = [])
    for (walk, hostname) in data
        current = DataFrame(time = 1:length(walk), y = walk, hostname = [hostname for i in 1:length(walk)])
        append!(all, current)
    end
    plot(all, x=:time, y=:y, color=:hostname, Geom.line)
end

Gadfly.push_theme(:dark)
render_results([(make_brownian(1, steps), "a"), (make_brownian(1, steps), "b")])
\end{minted}
\end{document}
